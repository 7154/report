\documentclass{jsarticle}
%\pagestyle{empty}%ページ非表示
%usepackage{fancyhdr}%ヘッダ
\usepackage[dvipdfmx]{graphicx}
\usepackage[top=30truemm,bottom=30truemm,left=20truemm,right=20truemm]{geometry}
\begin{document}
%ここから

%\lhead{左ヘッダ}
%\chead{中央ヘッダ}
%\rhead{右ヘッダ}

\title{UNIXの基礎とシェルプログラミング}
\author{筒居 稔}
\maketitle

%\part*{部}
%\chapter*{章}
%\subsection*{小節}
%\subsubsection*{少々節}
%改行\\
%\par
%改ページ\newpage

\section{目的}
UNIXコマンドを覚え、
bashによるシェルスクリプトの作成を行う。


\section{機器}

\begin{table}[hbtp]
 \caption{使用機器}
 \label{siyou}
 \centering
  \begin{tabular}{|c|c|}\hline
			実験機器名&ノートパソコン\\ \hline
  \end{tabular}
\end{table}

\section{内容}
UbuntuとWindowsのデュアルブート環境のため、
Ubuntuを起動して実験を行う。

%ここまで
\end{document}
