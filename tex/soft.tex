\documentclass[8pt,twocolumn]{jsarticle}
\pagestyle{empty}%ページ非表示
%usepackage{fancyhdr}%ヘッダ
\usepackage[dvipdfmx]{graphicx}
\usepackage[top=20truemm,bottom=0truemm,left=0truemm,right=0truemm]{geometry}
\begin{document}
\lhead{}
\chead{}
\rhead{5.I.11.筒居 稔}
\section{ソフトウェアの性質と開発の課題}
\subsection{ソフトウェアの特徴}
\subsubsection{ソフトウェアの構成}
プログラム,手法・技法,ドキュメントの
三つから構成される.
\subsubsection{ソフトウェア開発の特徴}
\begin{itemize}
		\item 実態がつかみにくい.
		\item 発工程に作業が集中する.
		\item 運用・保守期間が長い.
		\item 再利用が少ない.
\end{itemize}
\subsubsection{ソフトウェアの概念}
\begin{itemize}
		\item 要求仕様の満足度.
		\item 高い操作性.
		\item 適切な開発コストと開発期間.
		\item わかりやすさと保守性.
\end{itemize}
\subsection{ソフトウェアの分類}
\subsubsection{応用ソフトウェア}
\subsubsection*{業種別ソフトウェア}
金融業,製造業,流通業,サービス業.
\subsubsection*{業務別ソフトウェア}
販売,生産管理,購買,人事,経理等.
\subsubsection*{共通応用ソフトウェア}
表計算,統計処理,CAD,Web制作ツール等.
\subsubsection{ミドルウェア}
データベース管理,ネットワーク管理,GUI等.
\subsubsection{基本ソフトウェア}
OS,言語処理,サービスプログラム等.
\subsection{ソフトウェアのライフサイクル}
要求分析,外部設計,内部設計,プログラミングテスト,
保守・運用.
\subsection{ソフトウェア開発現場における課題}
\begin{itemize}
		\item 要求仕様決定の困難性
		\item 再利用の困難性
		\item プロジェクトトラブルの発生可能性
		\item ソフトウェア開発規模・工数見積もりの誤り
\end{itemize}
\section{ソフトウェア開発プロセス}
\begin{table}[htbp]
		\begin{tabular}{l|l}
				要求分析&要求仕様書\\
				外部設計&外部設計書(外部仕様書)\\
				内部設計&内部設計書(内部仕様書)\\
				プログラミング&ソースコード,プログラム仕様書\\
				テスト&テスト仕様書,成績書\\
				保守・運用&運用・保守マニュアル,操作説明書\\
		\end{tabular}
\end{table}
\subsection{プロセスモデル}
\subsubsection{ウォーターフォールモデル}
築時的なて銃ですすめる方法.
要求分析を明らかにしてから外部設計,
内部設計,プログラミング,テストを順番に行い,
次に保守・保守に移っていく.\\
特徴:築時的に開発を勧められ,
各段階の成果はドキュメントによって
引き継がれるので管理が容易.
各段階ごとの作業分担がしやすい.
大規模システムの開発に向いている.
開発方法として定着しており,開発要員の教育がしやすい.\\
問題点:要求定義の結果がコンピュータ上の
動作で確認されるまで長期間を要する.
作業中のある段階で遅れが生じると,
それ以降の段階に遅れが波及し,
次々にコウテイ遅れが生じる.
作業中の段階で上流段階の不具合が
見つかった場合,重龍にさかのぼって修正するのに
多くの量力要する.\\
問題点への対応作:要求定義の内容を
プロトタイピングで確認する.
サブシステムごとに分割開発できる場合には,
サブシステム単位にスパイラルモデルを用いる.\\
要求分析=運用テスト,外部設計=システムテスト,
内部設計=結合テスト,プログラミング=単体テスト
の相互で認証・検証の関係を
図で表しvモデル,vカーブと呼ぶ.
\subsubsection{プロトタイピングモデル}
プロトタイピングモデルは,要求分析の内容の
主要部分を試作し,ユーザとの
使用確認結果をフィードバックして,
プロトタイプを繰り返し修正しつつ要求内容の
確認を行うモデルである.\\
プロトタイピングモデルには
使い捨てプロトタイピングと進化型プロトタイピングがある.
進化型プロトタイピングモデルはプロトタイプを作成,
使用確認,機能追加を繰り返し実用化の間で持っていく.
\subsubsection{スパイラルモデル}
スパイラルモデルは,ウォータフォールモデルと,
プロトタイピングモデルの発展的進化を
組み合わせたモデルである.
このモデルは目標・対策・制約決定の区域,
対策評価の区域,開発検証の区域,
計画の区域,の四つの区域を渦巻状に開発を進めていく.
この四つの区域を一サイクルとして,
ソフトウェア開発をいくつかのサイクルを
繰り返して進めていく.
スパイラルモデルは,これらの区域を
繰り返し通ることにより,
仕様の確認,リスクの分析と回避,代替案の取り入れ
などを含めて開発を進めていくことが特徴である.
\section{要求分析}
ソフトウェア開発の初期段階で行われるもので,
何をこのソフトウェアで実現すべきかを明らかにする作業.
ユーザの要求を把握し,
満足してもらえる実現可能な要求モデルを作成し,
ソフトウェア設計者が正確に判断できる
要求仕様書にまとめること.\\
機能仕様:システムが実現する機能要件\\
非機能仕様:システムの持つべき性能,
保守性,安全性,信頼性,セキュリティ,相互運用性等で
品質特性に大きく関連する.
\subsection{要求分析における課題}
\begin{itemize}
				\item ユーザ要求のあいまいさ
				\item ユーザ要求の多様性
				\item ユーザと分析者間の相互理解の難しさ
				\item ユーザ要求の頻繁な変化
\end{itemize}
\subsection{要求分析の技法}
\begin{table}[htbp]
		\begin{tabular}{l|l}\\
				ユーザ要求の獲得&\shortstack{
						ユーザから直接,情報を収集整理し,
						\\知識として理解する作業}\\ \hline
				ユーザ要求の表現&\shortstack{
						得た知識を要求モデルあるは\\
						自然言語により表現し,\\
						機能やその他の仕様として記述する作業}\\ \hline
				ユーザ要求の妥当性確認&\shortstack{
						表現された要求仕様が\\ユーザの意図と
						あっているか確認,\\曖昧や抜け,
						矛盾がないか確認を行う作業}\\ \hline
		\end{tabular}
\end{table}
\subsection{ユーザ要求の獲得}
\begin{table}[htbp]
		\begin{tabular}{l|l}
				獲得する知識・情報&獲得の方法 \\ \hline
				解決スべき問題と関連情報&インタビューによる分析 \\
				組織構成&目的・目標の分析 \\
				実現システムの利害関係者&シナリオの利用 \\
				システム運用環境&プロトタイピング \\
				個別業務内容&業務関連文書の利用 \\
				問題領域の範囲& \\
		\end{tabular}
\end{table}
\subsection{ユーザ要求の妥当性確認}
\subsubsection{要求モデル検証の基準}
正当性:ユーザが望んでいること\\
完全性:重要な情報が欠落していないこと\\
一貫性:記述が矛盾していないこと\\
非曖昧性:二つ以上の意味に解釈されないこと\\
最小性:最小の記述量であること\\
検証可能性:テストできること\\
実現可能性:実現できること\\

\end{document}
