\documentclass{jsarticle}
%\pagestyle{empty}%ページ非表示
%usepackage{fancyhdr}%ヘッダ
\usepackage[dvipdfmx]{graphicx}
\usepackage[top=30truemm,bottom=30truemm,left=20truemm,right=20truemm]{geometry}
\usepackage{ulem}
\begin{document}
%ここから

%\lhead{左ヘッダ}
%\chead{中央ヘッダ}
%\rhead{右ヘッダ}

%\title{}
%\author{}
%\maketitle

%\part*{部}
%\chapter*{章}
%\section*{節}
%\subsection*{小節}
%\subsubsection*{少々節}
%改行\\
%\par
%改ページ\newpage

\section{目的}
LaTeXを用いて活版印刷に劣らない美しい
組版処理を行うための手法を取得し、
PDFファイルへの出力を行う。
また、LaTeXを用いると、複雑なフォント指定や
数式、画像、図、今日の出力が容易にでき、
目次、索引、ページ番号、文献参照なども
自動的に行うことができるようになる。

\section{使用機器}
\begin{table}[hbtp]
 \caption{使用機器}
 \label{siyou}
 \centering
  \begin{tabular}{|c|c|}\hline
			実験機器名&ノートパソコン\\ \hline
  \end{tabular}
\end{table}


\section{方法}
本実験ではレポートの作成が実験である。
結果の欄で
\TeX
の機能を紹介し実行、
成功したら\verb|\begin{verbatim}|と
\verb|\end{verbatim}|の間にコピーする。
使う機能は文章の改行や
フォントのサイズや種類の変更、
図表の表示、数式の作成等である。\par
本実験中使用する機能の内の
\verb|\sout|に必要なパッケージ
である\verb|ulem.sty|がなかったので、\\
\begin{center}
\verb|http://www.ctan.org/tex-archive/macros/latex/contrib/ulem|\\
\end{center}
からダウンロードして
\LaTeX
のディレクトリにstyファイルを配置した。

\section{結果}
\subsection{そのまま表示}
\TeX
で実行してほしくないコマンドを書くときに使う。

\begin{verbatim*}
\TeX
で書きました\\
\verb|\TeX|
で書きました
\begin{verbatim}
\TeXと\verb|\TeX|
\end{verbatim}
\end{verbatim*}

\noindent
\TeX
で書きました\\
\verb|\TeX|で書きました
\begin{verbatim}
\TeXと\verb|\TeX|
\end{verbatim}


\subsection{改行}

TeXで改行がしたい場合はそのまま
改行しても改行されない、
改行するためには改行コマンドを入れる。
改行のコマンドには\verb|\\|を使う。

\begin{verbatim}
ここから改行、\\ここまで改行、
改行されない。
\end{verbatim}

ここから改行、\\ここまで改行、
改行されない。

\subsection{改段落}

また改段落するときのコマンドもある。
改段落のコマンドは\verb|\par|を使い、
コマンドの後に改行をして改段落が行われる。

\begin{verbatim}
ここから改段落、\par
ここまで改段落、
改段落されない。
\end{verbatim}

ここから改段落、\par
ここまで改段落、
改段落されない。


\subsection{ロゴ}
TeX,LaTeXのロゴを表示する。

\begin{verbatim}
\TeX\par
\LaTeX
\end{verbatim}

\TeX\par
\LaTeX

\subsection{フォントの大きさ}
フォントのサイズを変換するコマンドもある。
以下に5つ例を記す。

\begin{verbatim}
\tiny 5pt \\
\small 9pt \\
\normalsize 10pt(標準) \\
\LARGE 17.28pt \\
\Huge 24.88pt \\
\end{verbatim}

{
\noindent
\tiny 5pt \\
\small 9pt \\
\normalsize 10pt(標準) \\
\LARGE 17.28pt \\
\Huge 24.88pt \\
}

\subsection{書体}
書体を変えることも可能、
日本語はゴシック体、明朝体。
英語はroman体、sans seriff体を例に記す。

\begin{verbatim}
\gt ABC いろは ゴシック体 \\
\mc ABC いろは 明朝体 \\
\rm ABC いろは roman \\
\sf ABC いろは sans seriff \\
\end{verbatim}

{
\noindent
\gt ABC いろは ゴシック体 \\
\mc ABC いろは 明朝体 \\
\rm ABC いろは roman \\
\sf ABC いろは sans seriff \\
}

\subsection{太字、イタリックアンダーライン、取り消し線}
文章の中で、強調したい部分などに使う。

\begin{verbatim}
\textbf {太字} \\
\textit {italic} \\
\underline {アンダーライン} \\
\sout {取り消し線} \\
\end{verbatim}

{
\noindent
\textbf {太字} \\
\textit {italic} \\
\underline {アンダーライン} \\
\sout {取り消し線} \\
}

\subsection{左寄せ、右寄せ、中央寄せ}
文字列の左寄せ、右寄せ、中央寄せを行うコマンド。

\begin{verbatim}
\begin{flushleft}
		左寄せ
\end{flushleft}
\begin{center}
		中央寄せ
\end{center}
\begin{flushright}
		右寄せ
\end{flushright}
\end{verbatim}

{
\begin{flushleft}
		左寄せ
\end{flushleft}
\begin{center}
		中央寄せ
\end{center}
\begin{flushright}
		右寄せ
\end{flushright}
}

\subsection{箇条書き}
箇条書きには記号付き、番号付き、見出し付きがある。

\begin{verbatim}
記号付き\\
\begin{itemize}
				\item item1
				\item item2
				\item item3
\end{itemize}
\noindent
番号付き\\
\begin{enumerate}
				\item item1
				\item item2
				\item item3
\end{enumerate}
\noindent
見出し付き\\
\begin{description}
		\item[その1] item1
		\item[その2] item2
		\item[その3] item3
\end{description}
\end{verbatim}

{
\noindent
記号付き\\
\begin{itemize}
				\item item1
				\item item2
				\item item3
\end{itemize}
\noindent
番号付き\\
\begin{enumerate}
				\item item1
				\item item2
				\item item3
\end{enumerate}
\noindent
見出し付き\\
\begin{description}
		\item[その1] item1
		\item[その2] item2
		\item[その3] item3
\end{description}
}


\subsubsection{画像}
画像ファイルの取得と表示もしてくれる、
画像ファイルは相対パスもしくは
絶対パスで指定する。\par
\verb|\includegraphics|の後に大カッコ\verb|"[]"|
内にオプションを記入する。
\verb|scale|オプションは画像の拡大縮小を行う、
記入する数値は拡大縮小率である。

\begin{verbatim}
\begin{figure}[htbp]
		\includegraphics[scale=0.6]{/home/minoru/Desktop/sample.eps}
\end{figure}
\end{verbatim}

\begin{figure}[htbp]
		\includegraphics[scale=0.6]{/home/minoru/Desktop/sample.eps}
\end{figure}

また、画像のサイズを変更することもできる。
\verb|widht|で幅、\verb|height|で高さを変える。
\verb|\fbox{}|で中カッコ内に画像を挿入したら、
画像を四角で囲う。
\verb|\centering|をbeginとendの間に記入したら、
中央寄せを行う。\par

\begin{verbatim}
\begin{figure}
		\centering
		\fbox{\includegraphics[width=5cm,height=10cm]{/home/minoru/Desktop/sample.eps}}
\end{figure}
\end{verbatim}

\begin{figure}
		\centering
		\fbox{\includegraphics[width=5cm,height=10cm]{/home/minoru/Desktop/sample.eps}}
\end{figure}

\verb|\centering|でなくても上で紹介したように、
\verb|begin{centering}|を用いることも可能で、
右寄せ、左寄せも可能である。

\begin{verbatim}
\begin{figure}[htbp]
		\begin{flushleft}
				\includegraphics[scale=0.1]{/home/minoru/Desktop/sample.eps}
		\end{flushleft}
\end{figure}
\begin{figure}[htbp]
		\begin{center}
				\includegraphics[scale=0.1]{/home/minoru/Desktop/sample.eps}
		\end{center}
\end{figure}
\begin{figure}[htbp]
		\begin{flushright}
				\includegraphics[scale=0.1]{/home/minoru/Desktop/sample.eps}
		\end{flushright}
\end{figure}
\end{verbatim}

\begin{figure}[htbp]
		\begin{flushleft}
				\includegraphics[scale=0.1]{/home/minoru/Desktop/sample.eps}
		\end{flushleft}
\end{figure}
\begin{figure}[htbp]
		\begin{center}
				\includegraphics[scale=0.1]{/home/minoru/Desktop/sample.eps}
		\end{center}
\end{figure}
\begin{figure}[htbp]
		\begin{flushright}
				\includegraphics[scale=0.1]{/home/minoru/Desktop/sample.eps}
		\end{flushright}
\end{figure}

\verb|\caption{}|の中カッコ内に図題を入れる、
図番は自動で割り振られる。
\verb|\label{}|では図のラベルを入れる、
\verb|\ref{}|でラベルを指定して、図番を参照する。\par

\begin{verbatim}
\begin{figure}[htbp]
		\centering
		\fbox{\includegraphics[scale=0.5]{/home/minoru/Desktop/sample.eps}}
		\caption{ねこです}
		\label{cat}
\end{figure}

図\ref{cat}はねこです。\\
\end{verbatim}

\begin{figure}[htbp]
		\centering
		\fbox{\includegraphics[scale=0.5]{/home/minoru/Desktop/sample.eps}}
		\caption{ねこです}
		\label{cat}
\end{figure}

図\ref{cat}はねこです。\\

\subsection{表}
表を作成してくれるコマンドです。
図と同じく\verb|\caption,\label|
で表題とラベルをつけて、参照することができる。\par

\begin{verbatim}
\begin{table}[htbp]
	\caption{ねこです}
	\label{nekodesu}
	\begin{tabular}{|r|c|l|} \hline
	右詰め&真ん中&左詰め \\ \hline
	\end{tabular}
\end{table}

表\ref{nekodesu}はねこです。\\
\end{verbatim}

\begin{table}[htbp]
	\centering
	\caption{ねこです}
	\label{nekodesu}
	\begin{tabular}{|r|c|l|} \hline
	右詰め&真ん中&左詰め \\ \hline
	\end{tabular}
\end{table}

表\ref{nekodesu}はねこです。\par

\subsection{minipage}

表は図と同じように\verb|\begin{figure}|
で囲うと図として扱われる。
\verb|\begin{figure}|内で\verb|\begin{minipage}|
を使うと、図や表を並べて表示することができる。\par

\begin{verbatim}
\begin{figure}
		\begin{minipage}{0.55555\hsize}
				\begin{center}
						\fbox{\includegraphics[scale=0.5]{/home/minoru/Desktop/sample.eps}}
						\caption{ねこです}
						\label{thiscat}
				\end{center}
		\end{minipage}
		\begin{minipage}{0.5\hsize}
				\begin{center}
						\caption{ねこです}
						\label{neko}
						\begin{tabular}{rcl}
								左詰め&真ん中&右詰め \\ \hline \hline
						\end{tabular}
				\end{center}
				\end{minipage}
\end{figure}
\end{verbatim}

\begin{figure}
		\begin{minipage}{0.5\hsize}
				\begin{center}
						\fbox{\includegraphics[scale=0.5]{/home/minoru/Desktop/sample.eps}}
						\caption{ねこです}
						\label{thiscat}
				\end{center}
		\end{minipage}
		\begin{minipage}{0.5\hsize}
				\begin{center}
						\caption{ねこです}
						\label{neko}
						\begin{tabular}{|r|c|l|} \hline
								左詰め&真ん中&右詰め \\ \hline
						\end{tabular}
				\end{center}
				\end{minipage}
\end{figure}


\subsection{数式}
文章中に数式を書く場合は\verb|$|で囲う。\\

\begin{verbatim}
これは式です$y=f(x)$。
\end{verbatim}

これは式です$y=f(x)$。\\
\par
別行に式を書く場合は\verb|\[|と\verb|\]|
で囲う。\\

\begin{verbatim}
これも式です\[y=ax~2\]これも式です。
\end{verbatim}

これも式です\[y=ax~2\]これも式です。\\
\par
数式番号を付ける場合には
\verb|\begin{equation}と\end{equation}|
で囲い、図番を参照するためのラベルは
囲った間に入れる。

\begin{verbatim}
式です
\begin{equation}
		y=5 \label{shiki}
\end{equation}
式\ref{shiki}は式です。
\end{verbatim}

式です
\begin{equation}
		y=5 \label{shiki}
\end{equation}
式\ref{shiki}は式です。\\

\subsection{分数}
分数を表示を表示するには、
\verb|\[\frac{分子}{分母}\]|
となる。

\begin{verbatim}
分数です\[\frac{1}{2}\]
\end{verbatim}

分数です\[\frac{1}{2}\]\\
\par
文章中に分数を表示するには
\verb|$\frack{分子}{分母}$|でできる。

\begin{verbatim}
分数です$\frac{1}{3}$
\end{verbatim}

分数です$\frac{1}{3}$\\

\subsection{微分}
微分の表示する方法を左に、
右に実行結果を記す。

\begin{minipage}{0.5\hsize}
		\begin{verbatim}
		\[y'=x^2+x+1\]
		\[y''=x^2+x+1\]
		\[\dot{y}=x^2+x+1\]
		\[\ddot{y}=x^2+x+1\]
		\[\frac{dy}{dx}=x^2+x+1\]
		\[\frac{d^2y}{dx^2}=x^2+x+1\]
		\[\frac{\partial f}{\partial x}=x^2+x+1\]
		\[\frac{\partial^2 f}{\partial x^2}=x^2+x+1\]
		\end{verbatim}
\end{minipage}
\begin{minipage}{0.5\hsize}
		\[y'=x^2+x+1\]
		\[y''=x^2+x+1\]
		\[\dot{y}=x^2+x+1\]
		\[\ddot{y}=x^2+x+1\]
		\[\frac{dy}{dx}=x^2+x+1\]
		\[\frac{d^2y}{dx^2}=x^2+x+1\]
		\[\frac{\partial f}{\partial x}=x^2+x+1\]
		\[\frac{\partial^2 f}{\partial x^2}=x^2+x+1\]
\end{minipage}

\subsection{積分}
積分は\verb|\int_{上限}^{下限} 被積分関数 dx|で表示する。

\begin{verbatim}
積分です$y=\int_{5}^{1} x^2 dx$ 
これも積分です\[y=\int_{5}^{1} x^2 dx\]
\end{verbatim}

積分です$y=\int_{5}^{1} x^2 dx$ 
これも積分です\[y=\int_{5}^{1} x^2 dx\]

\subsection{その他の数式}
\{による場合分け表示を下に記す。

\begin{verbatim}
\begin{equation}
		f(x)= \left\{
				\begin{array}{}
						1 (x=1のとき)\\
						0 (x≠1のとき)
				\end{array}
				\right.
\end{equation}
\end{verbatim}

\begin{equation}
		f(x)= \left\{
				\begin{array}{}
						1 (x=1のとき)\\
						0 (x≠1のとき)
				\end{array}
				\right.
\end{equation}\\
\par

上付き文字は\verb|^{上付き文字}|で表示し、
ドットは\verb|\dot|で表示する。

\begin{verbatim}
上付き文字$x^{2}$\\
ドット$\dot{x}$
\end{verbatim}

\noindent
上付き文字$x^{2}$\\
ドット$\dot{x}$\\
\par

下付き文字は\verb|_{下付き文字}|で表示する。

\begin{verbatim}
下付き文字$a_{n}$\\
こんなことも可能\[{}_{n}C_{k}\]
こんなことも\[{}^{i}_{j}T^{k}_{h}\]
\end{verbatim}

\noindent
下付き文字$a_{n}$\\
こんなことも可能\[{}_{n}C_{k}\]
こんなことも\[{}^{i}_{j}T^{k}_{h}\]
\par

根号は\verb|\sqrt[乗根]{数文字}|で表示する。

\begin{verbatim}
$\sqrt{x}\sqrt[3]{x}$
\end{verbatim}

\noindent
$\sqrt{x}\sqrt[3]{x}$\\
\par

\subsection{参考文献}
使用した参考文献一覧の作成は
\verb|\begin{thebibliography}|で行う。
\verb|\begin{thebibliography}|の後ろの
中カッコ内には数字を入れる、
参考文献の数が9以下なら9、
99以下なら99の様に書く。

\begin{verbatim}
ねこです\cite{neko}の著者ねこは、
とらのきもち\cite{tora}の定期購読をしている。

\begin{thebibliography}{9}
\bibitem{sanko} 参考文献名,著者名
\bibitem{neko} ねこです,ねこ
\bibitem{tora} とらのきもち,DJ KOYA
\bibitem{inu} いぬのきもち,DJ KOYA
\end{thebibliography}
\end{verbatim}

ねこです\cite{neko}の著者ねこは、
とらのきもち\cite{tora}の定期購読をしている。

\begin{thebibliography}{9}
\bibitem{sanko} 参考文献名,著者名
\bibitem{neko} ねこです,ねこ
\bibitem{tora} とらのきもち,DJ KOYA
\bibitem{inu} いぬのきもち,DJ KOYA
\end{thebibliography}

\section{考察}
本実験では
\LaTeX
を用いてレポートを作成したが、
レポートの作成に必要な機能が一通り揃っているため、
ストレス無くレポートの作成ができた。
今回のレポートで紹介した機能はまだ一部である、
他の機能を使うことにより論文記述や数学の問題作成、
楽譜の作成まで出来る。
これらの機能を組み合わせることにより、
レポートにとどまらず、思い思いの文書
(パンフレットやライブチケット等)を
作成することができると思う。



%ここまで
\end{document}
