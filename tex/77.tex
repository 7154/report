\documentclass{jsarticle}
%\pagestyle{empty}%ページ非表示
%usepackage{fancyhdr}%ヘッダ
\usepackage[dvipdfmx]{graphicx}
\usepackage[top=30truemm,bottom=30truemm,left=20truemm,right=20truemm]{geometry}
\usepackage{listings,jlisting}
\begin{document}
%ここから

%\lhead{左ヘッダ}
%\chead{中央ヘッダ}
%\rhead{右ヘッダ}

\title{PICを用いたLEDの点灯回路の政策とその外部割り込み実験}
\author{筒居 稔}
\maketitle

%\part*{部}
%\chapter*{章}
%\section*{節}
%\subsection*{小節}
%\subsubsection*{少々節}
%改行\\
%\par
%改ページ\newpage

\section{目的}
PICを用いてLED点灯回路を制作し、その外部割込み実験とその応用を行う。
Arduino等のマイコンによる電子制御回路を制作し、その動作確認実験を行う

\section{原理}
\subsection{マイコンの種類}
\subsubsection{PIC}
PICとは、ペリフェラル インターフェース コントローラーの略称であり、
マイクロチップ・テクノロジー社が製造している。
CPU・メモリ(RAM,ROM)、I/Oなどが1チップに全て収められている。
ROMに書き込まれたプログラムによって制御する。
プログラムコード用の内蔵メモリは、
古くはワンタイムROM、EPROM(紫外線消去)品があったが、
現在では大多数がフラッシュROM品となっている。

\subsubsection{H8}
H8は日立製作所(現在はルネサス エレクトロニクス)が
開発したマイクロコントローラである。
製品としてはCPUコアにROM、RAM、割り込みコントローラ、
タイマ、入出力ポート、シリアルコントローラ(SCI)、
A/Dコンバータ、D/Aコンバータ、DMA等が統合されたパッケージで販売される。
開発言語はC/C++/アセンブラである。

\subsubsection{AVR}
AVRとは、Atmel社が製造している、RISCベースの8および
32ビットマイクロコントローラ(制御用IC)製品群の総称である。
PIC同様に回路構成が簡単でCPU、メモリ(RAM、ROM)、I/O、
データ記憶用のEEPROM、クロック発信機、タイマ等が
1チップに収められており、書き込まれたプログラムによって制御される。\par

\subsection{PIC16F84A}
PIC16F84Aとはマイクロチップのフラッシュマイコンである。
1kバイトのメモリを搭載しており、
1000回程度プログラムを即時消去し、簡単に書き換えられる。
I/Oピン数は13であり、ピンごとに入出力設定が可能である。

\subsection{Arduinoとは}
Arduinoとは、Atmel AVRマイコンチップを実装した
基板と開発システムから構成される、
オープンソースハードウェアの一つである。
開発システムはArduinoホームページからダウンロードでき、
開発環境もオープンソース・マルチプラットフォーム対応なので
Mac OS XやLinuxでも開発できる。\\
開発環境ではC++風のArduono言語によりプログラムを開発する。

\subsection{部品について}
\subsubsection{セラミック発信機(セラロック)}
多結晶体である電圧セラミックスの機械的共振を利用した、
固有の周波数で発振する電子部品である。
主な用途は、マイクロプロセッサ等の
デジタル回路におけるクロック信号源である。

\subsubsection{発光ダイオード}
ダイオードの一種であり、順方向に電圧を
加えた際に発行する半導体素子である。

\section{使用機器}
\begin{table}[thbp]
		\caption{使用機器}
		\centering
		\label{kiki}
		\begin{tabular}{|l|c|r|}\hline
				使用機器名&型番&メーカー \\ \hline
				ノートPC&FMV-A 8260&FUJITSU \\ \hline
				デスクトップPC&CF-82&Pnasonic \\ \hline
				Arduino&& \\ \hline
		\end{tabular}
\end{table}

\section{方法}
\subsection{PIC実験}
PICマイコン回路を作成し、
サンプルプログラムを動作させて、
オリジナルの改良を加えて実行する
以下のプログラムをPICマイコンに書き込み動作させた。

\lstinputlisting[caption=サンプルプログラム,label=fuck]{/home/minoru/Desktop/zikken/first.c}
\lstinputlisting[caption=オリジナルプログラム,label=hell]{/home/minoru/Desktop/zikken/second.c}

\subsection{熱電対実験}
熱電対の出力をAD変換し、USB経由でパソコンで電圧を測定する。

\subsection{Arduin実験}
Arduinoを用いて応用実験を行う。
以下のプログラムをArduinoに書き込み実行した。
\lstinputlisting[caption=施錠確認プログラム,label=shit]{/home/minoru/Desktop/ard/77/77.ino}

\section{結果}
\subsection{PIC実験}
以下にサンプルプログラムと
オリジナルプログラムの結果を示す。
図\ref{sample}はサンプルプログラムの結果である。
スイッチを押すとすべてのLED
(写真の回路はハンダ付けが
うまくできなかったため一部のLED以外)
が点灯する。
スイッチから手を離すと消灯する。\par

図\ref{origin}、図\ref{origin2}はオリジナルプログラムの結果である。
スイッチを押すと任意のLEDが点灯し、
スイッチを離しても点灯し続ける。
スイッチ2を押すと点灯していたLEDが消灯する。


\begin{figure}[hbtp]
 \begin{minipage}{0.3\hsize}
 \begin{center}
	\fbox{\includegraphics[width=5cm]{/home/minoru/Desktop/zikken/npic/6276122747302.jpg}}
		\caption{サンプルプログラム結果}
		\label{sample}
 \end{center}
 \end{minipage}
 \begin{minipage}{0.3\hsize}
 \begin{center}
		\fbox{\includegraphics[width=5cm]{/home/minoru/Desktop/zikken/npic/6276122485853.jpg}}
		 \caption{オリジナルプログラム}
 \label{origin}
 \end{center}
 \end{minipage}
 \begin{minipage}{0.3\hsize}
 \begin{center}
		\fbox{\includegraphics[width=5cm]{/home/minoru/Desktop/zikken/npic/6276122612221.jpg}}
		 \caption{オリジナルプログラム2}
 \label{origin2}
 \end{center}
 \end{minipage}
\end{figure}

\subsection{熱電対実験}
\begin{figure}[Htbp]
		\centering
		\fbox{\includegraphics[width=14cm]{/home/minoru/Desktop/zikken/npic/1499113804266.jpg}}
		\caption{0度}
		\label{fir}
\end{figure}
\begin{figure}[Htbp]
		\centering
		\fbox{\includegraphics[width=14cm]{/home/minoru/Desktop/zikken/npic/1499113805609.jpg}}
		\caption{24度}
		\label{sec}
\end{figure}
\begin{figure}[Htbp]
		\centering
		\fbox{\includegraphics[width=14cm]{/home/minoru/Desktop/zikken/npic/1499113806803.jpg}}
		\caption{100度}
		\label{thr}
\end{figure}
測定した結果図\ref{fir}〜\ref{thr}のような結果が得られた。
図\ref{fir}は0度、
図\ref{sec}は24度
図\ref{thr}は100度の時の結果である。

\subsection{Arduino実験}
Arduinoにソースコード\ref{shit}を
書き込み実行した。\par

\begin{figure}[htbp]
		\begin{minipage}{0.5\hsize}
				\begin{center}
						\fbox{\includegraphics[width=5cm]{/home/minoru/Desktop/zikken/npic/6276123451596.jpg}}
						\caption{Arduino結果1}
						\label{ard1}
				\end{center}
		\end{minipage}
		\begin{minipage}{0.5\hsize}
				\begin{center}
						\fbox{\includegraphics[width=5cm]{/home/minoru/Desktop/zikken/npic/6276123002092.jpg}}
						\caption{Arduino結果2}
						\label{ard2}
				\end{center}
		\end{minipage}
\end{figure}

プログラムの内容はサーボモータにより
箱をロックするものである。\par
図\ref{ard1}は箱が閉じられている時、
図\ref{ard2}は箱が開いている時の結果である。

\section{考察}
\subsection{PIC}
ソースコード\ref{fuck}とソースコード\ref{hell}の
フローチャートを下に記す。

\begin{figure}[Htbp]
		\begin{minipage}{0.5\hsize}
				\begin{center}
						\fbox{\includegraphics[width=7cm]{/home/minoru/Desktop/firstprogram.png}}
						\caption{サンプルプログラムフローチャート}
						\label{flow1}
				\end{center}
		\end{minipage}
		\begin{minipage}{0.5\hsize}
				\begin{center}
						\fbox{\includegraphics[width=10cm]{/home/minoru/Desktop/secondprogram.png}}
						\caption{オリジナルプログラムフローチャート}
						\label{flow2}
				\end{center}
		\end{minipage}
\end{figure}

サンプルプログラムではスイッチ1"RA3"を読み取り、
入力が入ればwhile文でLEDを点灯させていた。\par
オリジナルプログラムではスイッチ1"RA3"の他に、
スイッチ2"RB0"を使い、
while文の内部でスイッチ1と2の状態
を読み取ることによりLEDの点灯、消灯を行う。

\subsection{熱電対}
図\ref{fir}〜\ref{thr}を比較すると。
0度(図\ref{fir})では電圧は0[V]一定で変化は無かった。
24度(図\ref{sec})と100度(\ref{thr})では互いに電圧が変化している。
24度では電圧の値が最大で約0.00007[V]、
最小で約0.00003[V]であるのに対して100度では、
最大最小ともに約0.002[V]と温度が高くなるに連れて電圧が高く変化している。
このことから温度が高くなると抵抗値が高くなると思われる。

\subsection{Arduino}
プログラム\ref{shit}では
実行時は赤色LEDを点灯し1秒間ブザーを鳴らす。
"number0fknocks"でスイッチの入力の回数を判断し、
もし3回入力が入れば緑色LEDを点灯しサーボモータを90度回し、
箱のlockを解除しシリアルモニタに"The box is unlocke!"
と出力する。
もう一度スイッチを入力するとサーボモータが90度回転し、
赤色LEDが点灯、1秒間ブザーが鳴りシリアルモニタに
"The box is locked!"と表示される。

%ここまで
\end{document}
