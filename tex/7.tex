\documentclass{jsarticle}
%\pagestyle{empty}%ページ非表示
%usepackage{fancyhdr}%ヘッダ
\usepackage[dvipdfmx]{graphicx}
\usepackage[top=30truemm,bottom=30truemm,left=20truemm,right=20truemm]{geometry}
\begin{document}
%ここから

%\lhead{左ヘッダ}
%\chead{中央ヘッダ}
%\rhead{右ヘッダ}

\title{PICを用いたLEDの点灯回路の政策とその外部割り込み実験}
\author{筒居 稔}
\maketitle

%\part*{部}
%\chapter*{章}
%\section*{節}
%\subsection*{小節}
%\subsubsection*{少々節}
%改行\\
%\par
%改ページ\newpage

\section{目的}
PICを用いてLED点灯回路を制作し、その外部割込み実験とその応用を行う。
Arduino等のマイコンによる電子制御回路を制作し、その動作確認実験を行う

\section{原理}
\subsection{マイコンの種類}
\subsubsection{PIC}
PICとは、ペリフェラル インターフェース コントローラーの略称であり、
マイクロチップ・テクノロジー社が製造している。
CPU・メモリ(RAM,ROM)、I/Oなどが1チップに全て収められている。
ROMに書き込まれたプログラムによって制御する。
プログラムコード用の内蔵メモリは、
古くはワンタイムROM、EPROM(紫外線消去)品があったが、
現在では大多数がフラッシュROM品となっている。

\subsubsection{H8}
H8は日立製作所(現在はルネサス エレクトロニクス)が
開発したマイクロコントローラである。
製品としてはCPUコアにROM、RAM、割り込みコントローラ、
タイマ、入出力ポート、シリアルコントローラ(SCI)、
A/Dコンバータ、D/Aコンバータ、DMA等が統合されたパッケージで販売される。
開発言語はC/C++/アセンブラである。

\subsubsection{AVR}
AVRとは、Atmel社が製造している、RISCベースの8および
32ビットマイクロコントローラ(制御用IC)製品群の総称である。
PIC同様に回路構成が簡単でCPU、メモリ(RAM、ROM)、I/O、
データ記憶用のEEPROM、クロック発信機、タイマ等が
1チップに収められており、書き込まれたプログラムによって制御される。\par

\subsection{PIC16F84A}
PIC16F84Aとはマイクロチップのフラッシュマイコンである。
1kバイトのメモリを搭載しており、
1000回程度プログラムを即時消去し、簡単に書き換えられる。
I/Oピン数は13であり、ピンごとに入出力設定が可能である。

\subsection{Arduinoとは}
Arduinoとは、Atmel AVRマイコンチップを実装した
基板と開発システムから構成される、
オープンソースハードウェアの一つである。
開発システムはArduinoホームページからダウンロードでき、
開発環境もオープンソース・マルチプラットフォーム対応なので
Mac OS XやLinuxでも開発できる。\\
開発環境ではC++風のArduono言語によりプログラムを開発する。

\subsection{部品について}
\subsubsection{セラミック発信機(セラロック)}
多結晶体である電圧セラミックスの機械的共振を利用した、
固有の周波数で発振する電子部品である。
主な用途は、マイクロプロセッサ等の
デジタル回路におけるクロック信号源である。

\subsubsection{発光ダイオード}
ダイオードの一種であり、順方向に電圧を
加えた際に発行する半導体素子である。


%ここまで
\end{document}
