\documentclass{jsarticle}
%\pagestyle{empty}%ページ非表示
\usepackage{fancyhdr}%ヘッダ
\usepackage[dvipdfmx]{graphicx}
\usepackage[top=30truemm,bottom=30truemm,left=20truemm,right=20truemm]{geometry}
\begin{document}
%ここから

\lhead{}
\chead{}
\rhead[実験 No.8-5I-No.11-筒居 稔]{実験 No.8-5I-No.11-筒居 稔}

\title{回路シミュレータによる電子回路の実験}
\author{筒居 稔}
\maketitle

\section{目的}
Spice系回路シミュレータである
μCAP9を使って、
回路シミュレーションおよび
設計した回路の制作および特性を測定する。
波形整形回路の過度解析や
増幅回路の交流解析を行う。
\section{原理}
\subsection{回路シュミレータ}
回路シミュレータとは、回路図または回路素子をつないだ
情報(ネットリスト)から、
その回路の電圧電流を計算してくれるツール。
%\part*{部}
%\chapter*{章}
%\subsection*{小節}
%\subsubsection*{少々節}
%改行\\
%\par
%改ページ\newpage

%ここまで
\end{document}
