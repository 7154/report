\documentclass{jsarticle}
%\pagestyle{empty}%ページ非表示
\usepackage{fancyhdr}%ヘッダ
\pagestyle{fancy}
\usepackage[dvipdfmx]{graphicx}
\usepackage[top=30truemm,bottom=30truemm,left=20truemm,right=20truemm]{geometry}
\begin{document}
%ここから
\rhead[5I-11:筒居 稔]{5I-11:筒居 稔}


\section{目的}
Spice系回路シミュレータである
μCAP9を使って、
回路シミュレーションおよび
設計した回路の制作および特性を測定する。
波形整形回路の過度解析や
増幅回路の交流解析を行う。

\section{原理}
\subsection{回路シミュレートを行うことのメリット}
\begin{itemize}
				\item 部品、実験ボード、測定器などがいらないため、コスト削減できる。
				\item 各手配等、工程を省けるため、時間の短縮になる。
				\item 実測不可能なものも測ることができる。
				\item 感電等の心配が無いため安全である。
\end{itemize}

\subsection{回路シミュレーションの限界}
\begin{itemize}
				\item 大体の回路規模は、2万〜3万素子程度が限界、これ以上規模が大きくなると、意味のある時間で処理が終わらない。
\end{itemize}

\section{方法}
\begin{enumerate}
				\item 回路を回路シミュレータに書き込み実行。
				\item 得られた波形データを保存。
\end{enumerate}

\section{使用機器}
表\ref{use}に使用機器を示す。
\begin{table}[htbp]
		\centering
		\caption{使用機器}
		\label{use}
		\begin{tabular}{|c|c|c|c|}\hline
				名称&メーカ&型番&個数 \\ \hline
				波形生成器&GW INSTEK&GAG-810&1 \\ \hline
				電源&GW INSTEK&GPS-18500&1 \\ \hline
				オシロスコープ&GW INSTEK&GDS-1052-U&1 \\ \hline
				ノートパソコン&DELL&& \\ \hline
		\end{tabular}
\end{table}

\section{結果}
実験方法の手順通りに回路をシミュレータで作成し、
その波形のデータを保存した。
保存した結果を下に記す。

\subsection{回路d}
図\ref{dskairo}は回路dの回路図であり、
図\ref{dshakei}はその波形である。
\begin{figure}[htpb]
		\begin{minipage}{0.45\hsize}
				\centering
				\fbox{\includegraphics[scale=1]{/home/minoru/Desktop/zikken/zik8/dskairo.png}}
				\caption{回路d回路図}
				\label{dskairo}
		\end{minipage}
		\begin{minipage}{0.3\hsize}
				\centering
				\fbox{\includegraphics[scale=0.3]{/home/minoru/Desktop/zikken/zik8/dshakei.png}}
				\caption{回路d波形}
				\label{dshakei}
		\end{minipage}
\end{figure}

\subsection{無安定バイブレータ}
図\ref{mbkairo}は無安定バイブレータの回路図であり、
図\ref{mbhakei}はその波形である。
\begin{figure}[htpb]
		\begin{minipage}{0.45\hsize}
				\centering
			\fbox{\includegraphics[scale=0.4]{/home/minoru/Desktop/zikken/zik8/mbkairo.jpg}}
				\caption{無安定バイブレータ回路図}
				\label{mbkairo}
		\end{minipage}
		\begin{minipage}{0.3\hsize}
				\centering
				\fbox{\includegraphics[scale=0.3]{/home/minoru/Desktop/zikken/zik8/mbhakei.jpg}}
				\caption{無安定バイブレータ波形}
				\label{mbhakei}
		\end{minipage}
\end{figure}

\subsection{シュミットトリガ}
図\ref{smkairo}はシュミットトリガの回路図であり、
図\ref{smhakei}はその波形である。
\begin{figure}[htpb]
		\begin{minipage}{0.45\hsize}
				\centering
				\fbox{\includegraphics[scale=0.4]{/home/minoru/Desktop/zikken/zik8/smkairo.png}}
				\caption{シュミットトリガ回路図}
				\label{smkairo}
		\end{minipage}
		\begin{minipage}{0.3\hsize}
				\centering
				\fbox{\includegraphics[scale=0.3]{/home/minoru/Desktop/zikken/zik8/smhakei.png}}
				\caption{シュミットトリガ波形}
				\label{smhakei}
		\end{minipage}
\end{figure}

\subsection{CR位相波形回路}
図\ref{crkairo}はCR位相波形回路の回路図であり、
図\ref{crhakei}はその波形である。
\begin{figure}[htpb]
				\centering
				\fbox{\includegraphics[scale=0.5]{/home/minoru/Desktop/zikken/zik8/crkairo.png}}
				\caption{CR位相波形回路回路図}
				\label{crkairo}
\end{figure}
\begin{figure}[htpb]
				\centering
				\fbox{\includegraphics[scale=0.3]{/home/minoru/Desktop/zikken/zik8/crhakei.png}}
				\caption{CR位相波形回路波形}
				\label{crhakei}
\end{figure}

\section{考察}
以下に実際に回路を作成し測定した波形を記す。
\begin{figure}[htbp]
		\centering
		\fbox{\includegraphics[scale=0.2]{/home/minoru/Desktop/zikken/npic/38816.jpg}}
		\caption{回路d}
		\label{dskairor}
\end{figure}

\begin{figure}[htbp]
		\centering
		\fbox{\includegraphics[scale=0.2]{/home/minoru/Desktop/zikken/zik8/38815.jpg}}
		\caption{無安定バイブレータ}
		\label{mbkairou}
\end{figure}

回路dはローパスフィルタである、
論理値では電圧が2.5[V]低下するはずが、
シミュレーションでは1.9[V]しか低下しなかった。
これはシミュレーションの設定を誤ってしまい
入力が直流になってしまったためである。
そのため実機の測定とは異なる結果になった。
実測では2.0[V]の電圧の低下が測定できた。\\

無安定バイブレータの周期の理論値は
\[周期=2(0.69CR)=2(0.69*180*10^{3}*200*10^{-12})=4.968*10^{-5}[s]\]
となるが、シミュレーションの結果では$43*10^{-6}[s]$となり、
実測では$55*10^{-6}[s]$となった。
シミュレーション、実測共に理論値と誤差が生じたが、
実測値がシミュレーションよりも理論値との誤差が少なく、
シミュレーションよりも制度が良いことがわかった。\\

シュミットトリガ回路はシミュレーションの結果、
3.8VでON,OFFが切り替わる。理論値では2.5V低下であるが、
実際には1.9Vであった。これは
ダイオードの立ち上がり電圧が原因と思われる。\\

CR位相波形回路は理論値では下記の式の結果300[Hz]となり、
\[f=\frac{1}{2}\pi\sqrt{6CR}=\frac{1}{2}\pi\sqrt{6*0.12*10^{-6}*1.8*10^{3}}≒300[Hz]\]
シミュレーション結果は40[ms]で
10回振動したため周波数は250[Hz]であった、
これは計測時にグラフに目盛りを表示していなかったため、
読み取り時に人為的な誤差が生じたと思われる。

%\part*{部}
%\chapter*{章}
%\subsection*{小節}
%\subsubsection*{少々節}
%改行\\
%\par
%改ページ\newpage

%ここまで
\end{document}
