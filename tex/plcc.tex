\documentclass{jsarticle}
%\pagestyle{empty}%ページ非表示
%usepackage{fancyhdr}%ヘッダ
\usepackage[dvipdfmx]{graphicx}
\usepackage[top=30truemm,bottom=30truemm,left=20truemm,right=20truemm]{geometry}
\begin{document}
%ここから

\title{PLCによるシーケンス制御}
\author{筒居 稔}
\maketitle

%\lhead{左ヘッダ}
%\chead{中央ヘッダ}
%\rhead{右ヘッダ}

%\part*{部}
%\chapter*{章}
\section{実験目的}
シーケンス制御とは、あらかじめ定められた動作順序にしたがって、機械や装置を制御するこ
とをいう。シーケンス制御には有接点リレー方式、無接点リレー方式、マイクロコンピュータ方
式などがある。本実験では、シーケンス図やラダー図を理解し、マイクロコンピュータを利用し
た制御装置であるプログラマブル・ロジック・コントローラを使用した制御方法を習得する。
%\subsection*{小節}
%\subsubsection*{少々節}
%改行\\
%\par
%改ページ\newpage
\section{実験機器}
\begin{table}[hbtp]
 \caption{使用機器}
 \centering
  \begin{tabular}{|c|c|c|}\hline
  機器名&メーカー&製品名\\ \hline\hline
  PLC&三菱&FX2-32MT-SIM\\ \hline
  \end{tabular}

 \caption{PLCの構成要素}
 \centering
  \begin{tabular}{|c||c|}\hline
  入力リレー&x0〜x7\\ \hline
  出力リレー&y20〜y27\\ \hline
  タイマ(100ms)&t0~t199\\ \hline
  タイマ(10ms)&t200~t245\\ \hline
  \end{tabular}
\end{table}
\section{原理}
PLC(シーケンサ)内にはリレー、タイマ、カウンタ等の機能が組み込まれており、
マイクロコンピュータを利用する電子制御装置である。
PLCと入出力機器はスイッチや各種センサなどの機器を入力とし、
ランプやブザー、電力開閉器等の機器を出力して制御する。
\section{方法}
AND回路,OR回路,自己保持回路について,
真理値表,タイムチャート,シーケンス図,ラーダ図を作成する。\par
先行優先回路,新入力優先回路,フリッカ回路について,
タイムチャート,シーケンス図,ラダー図を作成する。\par
それぞれラダー図を元にプログラムを作成し実行する。\newpage
\section{結果}
実験方法の手順に従い、真理地表、タイムチャート、
シーケンス図、ラダー図を作成し、
ラダー図を元にプログラムを作成、
実行して正しく動作することをを確認した。

\subsection{AND回路}
\begin{table}[hbtp]
 \caption{AND回路真理値表}
 \centering
 \begin{tabular}{|c|c|c|}\hline
 BS1&BS2&SL1\\ \hline\hline
 0&0&0\\ \hline
 0&1&0\\ \hline
 1&0&0\\ \hline
 1&1&1\\ \hline
 \end{tabular}
\end{table}

\begin{figure}[hbtp]
 \centering
 \includegraphics[scale=0.5]{/home/minoru/Desktop/sp.eps}
\end{figure}
\begin{figure}[hbtp]
 \caption{AND回路シーケンス図}
 \includegraphics[scale=0.5]{/home/minoru/Desktop/sp.eps}
 \caption{AND回路タイムチャート}
\end{figure}
\begin{figure}[hbtp]
 \includegraphics[scale=0.5]{/home/minoru/Desktop/sp.eps}
 \caption{AND回路ラダー図}
\end{figure}
\begin{figure}[hbtp]
\centering
\begin{tabular}{rl}
LD&X1\\
AND&X2\\
OUT&Y21\\
END&\\
\end{tabular}
\caption{AND回路プログラム}
\end{figure}
\newpage
\subsection{OR回路}
\begin{table}[hbtp]
 \centering
 \caption{OR回路真理値表}
 \begin{tabular}{|c|c|c|}\hline
 BS1&BS2&SL1\\ \hline\hline
 0&0&0\\ \hline
 0&1&1\\ \hline
 1&0&1\\ \hline
 1&1&1\\ \hline
 \end{tabular}
\end{table}

\begin{figure}[hbtp]
 \centering
 \includegraphics[scale=0.5]{/home/minoru/Desktop/sp.eps}
 \caption{OR回路シーケンス図}
\end{figure}

\begin{figure}[hbtp]
 \includegraphics[scale=0.5]{/home/minoru/Desktop/sp.eps}
 \caption{OR回路タイムチャート}
\end{figure}
\begin{figure}[hbtp]
 \includegraphics[scale=0.5]{/home/minoru/Desktop/sp.eps}
 \caption{OR回路ラダー図}
\end{figure}
\begin{figure}[htbp]
\centering
\begin{tabular}{rl}
LD&X1\\
OR&X2\\
OUT&Y21\\
END&\\
\end{tabular}
\caption{OR回路プログラム}
\end{figure}
\newpage
\subsection{自己保持回路}
\begin{table}[hbtp]
 \centering
 \caption{自己保持回路真理値表}
 \begin{tabular}{|c|c|}\hline
 BS1&SL1\\ \hline\hline
 0&直前の状態を保持\\ \hline
 1&1\\ \hline
 \end{tabular}
\end{table}

\begin{figure}[hbtp]
 \includegraphics[scale=0.5]{/home/minoru/Desktop/sp.eps}
 \caption{自己保持回路シーケンス図}
\end{figure}
\begin{figure}[hbtp]
 \includegraphics[scale=0.5]{/home/minoru/Desktop/sp.eps}
 \caption{自己保持回路タイムチャート}
\end{figure}
\begin{figure}[hbtp]
 \includegraphics[scale=0.5]{/home/minoru/Desktop/sp.eps}
 \caption{自己保持回路ラダー図}
\end{figure}
\begin{figure}[htbp]
\centering
\begin{tabular}{rl}
LD&X1\\
OR&Y21\\
OUT&Y21\\
END&\\
\end{tabular}
\caption{自己保持回路プログラム}
\end{figure}
\newpage
\subsection{先行優先回路}
\begin{figure}[htbp]
 \centering
 \includegraphics[scale=0.5]{/home/minoru/Desktop/sp.eps}
 \caption{先行優先回路シーケンス図}
\end{figure}
\begin{figure}[htbp]
 \includegraphics[scale=0.5]{/home/minoru/Desktop/sp.eps}
 \caption{先行優先回路タイムチャート}
\end{figure}
\begin{figure}[htbp]
 \includegraphics[scale=0.5]{/home/minoru/Desktop/sp.eps}
 \caption{先行優先回路ラダー図}
\end{figure}
\begin{figure}[htbp]
\centering
\begin{tabular}{rl}
LD&X1\\
OR&Y21\\
ANI&Y22\\
OUT&Y21\\
LD&X2\\
OR&Y22\\
ANI&Y21\\
OUT&Y22\\
END&\\
\end{tabular}
\caption{先行優先回路プログラム}
\end{figure}

\subsection{新入力優先回路}
\begin{figure}[htbp]
 \centering
 \includegraphics[scale=0.5]{/home/minoru/Desktop/sp.eps}
 \caption{新入力優先回路シーケンス図}
 \includegraphics[scale=0.5]{/home/minoru/Desktop/sp.eps}
 \caption{新入力優先回路タイムチャート}
 \includegraphics[scale=0.5]{/home/minoru/Desktop/sp.eps}
 \caption{新入力優先回路ラダー図}
\end{figure}
\begin{figure}[htbp]
\centering
\begin{tabular}{rl}
LD&X1\\
LD&Y21\\
ANI&Y22\\
ORB&\\
OUT&Y21\\
LD&X2\\
LD&Y22\\
ANI&Y21\\
ORB&\\
OUT&Y22\\
END&\\
\end{tabular}
\caption{新入力優先回路プログラム}
\end{figure}
\newpage
\subsection{フリッカ回路}
\begin{figure}[htbp]
 \centering
 \includegraphics[scale=0.5]{/home/minoru/Desktop/sp.eps}
 \caption{フリッカ回路シーケンス図}
\end{figure}
\begin{figure}[htbp]
 \includegraphics[scale=0.5]{/home/minoru/Desktop/sp.eps}
 \caption{フリッカ回路タイムチャート}
\end{figure}
\begin{figure}[htbp]
 \includegraphics[scale=0.5]{/home/minoru/Desktop/sp.eps}
 \caption{フリッカ回路ラダー図}
\end{figure}
\begin{figure}[htbp]
\centering
\begin{tabular}{rl}
LD&X1\\
ANI&T2\\
OUT&T1\\
K&005\\
LD&T1\\
OUT&Y21\\
OUT&T2\\
K&010\\
END&\\
\end{tabular}
\caption{フリッカ回路プログラム}
\end{figure}
%ここまで
\end{document}
